\documentclass[12pt,letterpaper]{article}
\usepackage[parfill]{parskip}

\begin{document}

\section*{CptS-411 | Homework-01 }
\subsection*{Charles Norden, \#011606177 }

\subsection*{PROBLEM 1:}
Consider a program whose sequential execution time is 100 days. Assume that the program can be parallelized and the performance scales linearly in terms of number of processors. Write an expression to compute the parallel execution time in terms of P, where P is the number of parallel processors. (1 mark)

\hline

Given the 100-day execution time, that is our upperbound for time complexity. The amount of time we can parallelize is at most equal to this time. The Speedup S is:

\[ S = \frac{p}{p * t_{s} + t_{p}} \]

Thus, parallel time \(t_{p}\) is as follows:

\[ t_{p} = \frac{p}{S} - p * t_{s} \]

\[ t_{p} = \frac{p}{S} - p * 100 \]

where \(t_{s} = 0\) days, assuming that the entirety of the program is parallelizable.

\pagebreak


\subsection*{PROBLEM 2:}
Consider a program whose sequential execution time is 100 days. Part A of this program cannot be parallelized and accounts for 40\% of total execution time. Part B of this program can be linearly be parallelized and the performance scales linearly in terms of number of processors. Part B accounts for 60\% of total execution time. Write an expression to compute the parallel execution time in terms of P, where P is the number of parallel processors. (1 mark)

\hline

Given the 100-day execution time, that is our upperbound for time complexity. The amount of time we can parallelize is at most equal to \(60%\) of this amount. The Speedup S is:

\[ S = \frac{p}{p * t_{s} + t_{p}} \]

Thus, parallel time \(t_{p}\) is as follows:

\[ t_{p} = \frac{p}{S} - p * t_{s} \]

\[ t_{p} = \frac{p}{S} - p * 40 \]

where \(t_{s} = 40\) days.

\pagebreak


\subsection*{PROBLEM 3:}
Write all data dependences in the following codes (concise representation is fine for 
part B) (2 marks)

\hline

a. \(d\) is dependent on \(a\) and \(b\). \(c\) is dependent on \(a\) and \(b\).

b. 

\pagebreak


\subsection*{PROBLEM 4:}
Write a parallel program to add two 3-D arrays A[100][100][100] and B[100][100][100]. 
Report the execution time for num threads = 1, 2, 4, 8, 16, and 32. How do you explain 
the observed performance trend as a function of number of threads( a simple textual 
explain is fine). (3 marks)

\hline

\end{document}
